 \documentclass{article}
 \usepackage{graphicx} 
 \title{My Document}
 \author{A. Student}
 \date{January 1, 2011}
 \begin{document}
 \maketitle
 \begin{enumerate}
 \item Here is some \textbf{boldfaced} text.
 \item Here is some \emph{emphasized} text.
 \end{enumerate}
 
 {\Large Hello!}
  $\lim_{x \rightarrow \infty}$
   \begin{enumerate}
 \item Suppose that $x=137$.
 \item Let$n=3$. Then$n^2+1=10$.
 \item Thecurve $y= \sqrt{x}$, where$x \geq 0$,is concave downward.
 \item If$\sin \theta = 0$and $0\leq \theta <2\pi$,
 then $\theta=0$ or $\theta=\pi$.
 \item Itis notalways true that
 \[\frac{a+b}{c+d}=\frac{a}{c}+\frac{b}{d}.\]
 \end{enumerate}
 \[
 \left[
 \begin{array}{ccc}
 a & b & c\\
 d & e & f\\
 g & h & i
 \end{array}
 \right]
 \]
 \begin{eqnarray*}
 1! + x2
 2! + x3
 3! +
 1! + (−1)2
 1! + 1
 2! − 1
 3! +
 2! + (−1)3
 3! +
 e^x & = & \frac{x^0}{0!}+\frac{x^1}{1!}
 +\frac{x^2}{2!}+\frac{x^3}{3!}+\cdots\\
 e^{-1} & = & \frac{{(-1)}^0}{0!}+\frac{{(-1)}^1}{1!}
 +\frac{{(-1)}^2}{2!}+\frac{{(-1)}^3}{3!}+\cdots\\
 & = & \frac{1}{0!}-\frac{1}{1!}+\frac{1}{2!}-\frac{1}{3!}+\cdots
 \end{eqnarray*}
  \begin{enumerate}
 \item Let $\mathbf{x}=(x_1,\ldots,x_n)$,
 where the $x_i$ are nonnegative real numbers.
 Set
 \[
 M_r(\mathbf{x}) = \left(\frac{x_1^r+x_2^r
 +\cdots+x_n^r}{n}\right)^{1/r},
 \; \; r \in \mathbf{R} \setminus \{0\},
 \]
 and
 \[
 M_0(\mathbf{x})=\left( x_1 x_2 \ldots x_n \right)^{1/n}.
 \]
 We call $M_r(\mathbf{x})$ the \emph{$r$th power mean}
 of $\mathbf{x}$.
 Claim:
 \[
 \lim_{r \rightarrow 0} M_r(\mathbf{x}) =
 M_0(\mathbf{x}).
 \]
 \item Define
 \[
 V_n=
 \left[
 \begin{array}{ccccc}
 1 & 1 & 1 & \ldots & 1\\
 x_1 & x_2 & x_3 & \ldots & x_n\\
 x_1^2 & x_2^2 & x_3^2 & \ldots & x_n^2\\
 \vdots & \vdots & \vdots & \ddots & \vdots\\
 x_1^{n-1} & x_2^{n-1} & x_3^{n-1} & \ldots & x_n^{n-1}
 \end{array}
 \right].
 \]
 We call $V_n$ the \emph{Vandermonde matrix} of order $n$.
 Claim:
 \[
 \det V_n = \prod_{1 \leq i < j \leq n}(x_j-x_i).
 \]
 
  \begin{picture}(250,75)
 %draw triangle
 \put(15,10){\line(1,0){50}}
 \put(65,10){\line(0,1){50}}
 \put(65,60){\line(-1,-1){50}}
 % draw square
 \put(100,10){\line(1,0){50}}
 \put(150,10){\line(0,1){50}}
 \put(150,60){\line(-1,0){50}}
 \put(100,60){\line(0,-1){50}}
 % draw circle
 \put(200,35){\circle{40}}
 \end{picture}
 \end{enumerate}

 \end{document}